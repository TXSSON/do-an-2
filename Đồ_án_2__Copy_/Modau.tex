\chapter*{Tóm tắt nội dung}
%\addcontentsline{toc}{chapter}{Tóm tắt nội dung}
%Nội dung đồ án nghiên cứu và xây dựng quy trình CI/CD trên nhiều nền tảng khác nhau. Được phát triển dựa trên mô hình được đưa ra trong một bài báo công bố khoa học \textit{Privacy preserving verifiable federated learning scheme using blockchain and homomorphic encryption} \cite{1.3} được công bố trong cuộc thi \textit{Applied Soft Computing 167}. Bài báo này giới thiệu một mô hình học liên kết có thể xác minh và bảo vệ quyền riêng tư mới (PPVFL – Privacy-Preserving Verifiable Federated Learning) tích hợp công nghệ blockchain và mã hóa đồng hình (homomorphic encryption) nhằm giải quyết các thách thức quan trọng trong học máy. Mô hình đề xuất đảm bảo quyền riêng tư dữ liệu, tính toàn vẹn, khả năng xác minh, bảo mật mạnh mẽ và hiệu quả trong việc bảo vệ quyền riêng tư của người dùng.
%
%\vfill
%\begin{tabular}{c c}
%      \hspace{6.5cm}  &  \textit{Hà Nội, ngày 13 tháng 06 năm 2025}\\
%       &Sinh viên\\
%       \vspace{2cm}\\
%       & \textbf{Bùi Anh Tú}
%\end{tabular}
%\vfill

\chapter*{Mở đầu}
\addcontentsline{toc}{chapter}{Mở đầu}

Trong kỷ nguyên chuyển đổi số, nơi tốc độ phát triển và chất lượng sản phẩm là yếu tố sống còn, khả năng triển khai phần mềm nhanh chóng, liên tục và đáng tin cậy đã trở thành ưu tiên hàng đầu của mọi tổ chức. Phương pháp phát triển truyền thống với các quy trình tích hợp và phân phối thủ công không còn đáp ứng được yêu cầu về tốc độ thị trường. Chính vì vậy, Tích hợp Liên tục và Phân phối/Triển khai Liên tục (CI/CD) đã nổi lên như một giải pháp kỹ thuật cốt lõi.

Về mặt lịch sử, khái niệm Tích hợp Liên tục (CI) được giới thiệu lần đầu tiên bởi Grady Booch vào năm 1991, nhưng đã được phổ biến rộng rãi và đưa vào thực tiễn phát triển phần mềm hiện đại bởi Kent Beck vào cuối những năm 1990. Tiếp nối CI, khái niệm Phân phối Liên tục (Continuous Delivery – CD) đã được hệ thống hóa và làm rõ trong cuốn sách cùng tên (xuất bản năm 2010) của Jez Humble và David Farley, trở thành nền tảng quan trọng trong văn hóa DevOps ngày nay.

CI/CD tự động hóa toàn bộ vòng đời phát triển phần mềm, từ khi các lập trình viên tích hợp mã nguồn (CI) đến khi ứng dụng được chuyển giao hoặc triển khai tới môi trường sản xuất (CD). Tầm quan trọng của CI/CD được chứng minh rõ rệt qua các số liệu: theo khảo sát của GitLab (2024), các nhóm phát triển phần mềm áp dụng CI/CD cho thấy tỷ lệ thành công khi triển khai tăng 42\% so với các nhóm không áp dụng phương pháp này, đồng thời giúp giảm thiểu đáng kể thời gian xử lý lỗi (MTTR), một chỉ số then chốt về khả năng bảo trì hệ thống. Báo cáo đồ án 2 sẽ đi sâu vào nghiên cứu và triển khai quy trình CI/CD trên nhiều nền tảng. 
%Chúng tôi sẽ đánh giá các chiến lược và công cụ cần thiết để duy trì tính nhất quán, hiệu suất và bảo mật cho pipeline CI/CD khi triển khai các ứng dụng trên các môi trường đa dạng, từ hạ tầng đám mây (Cloud) đến các hệ thống vật lý (On-premise) hay nền tảng di động (Mobile).

%Nguồn tham khảo số liệu/lịch sử:
%
%Grady Booch & Kent Beck: Nguồn gốc của Continuous Integration (CI).
%
%Jez Humble & David Farley: Tác giả cuốn sách "Continuous Delivery" (2010), hệ thống hóa Continuous Delivery (CD).
%
%Tỷ lệ thành công khi triển khai tăng 42\%: Thống kê từ GitLab (theo báo cáo/nghiên cứu về CI/CD).
%
%Giảm thiểu thời gian xử lý lỗi (MTTR): Lợi ích cốt lõi được công nhận rộng rãi của CI/CD (Tham khảo các tài liệu và nghiên cứu về DevOps/CI/CD).

Nội dung đồ án 2 được tổ chức thành các chương sau:

\begin{itemize}
	\item Chương 1. Giới thiệu chung: Trình bày tổng quát về CI/CD (Tích hợp và Phân phối/Triển khai Liên tục), bao gồm khái niệm, ưu điểm, hạn chế, và cách thức hoạt động cơ bản của quy trình.
	
	\item Chương 2. Giới thiệu các nền tảng và công cụ triển khai CI/CD: Giới thiệu và so sánh các nền tảng CI/CD cốt lõi được sử dụng: GitHub, GitLab, và Jenkins. Đồng thời, giới thiệu các công cụ hỗ trợ cần thiết khác như VMware Workstation Pro, Nginx, và SonarQube.
	
	\item Chương 3. Xây dựng hệ thống server ảo: Trình bày chi tiết việc xây dựng môi trường server ảo (VMs) và mô tả Kiến trúc hệ thống cụ thể cho việc triển khai CI/CD trên từng nền tảng (GitHub, GitLab và Jenkins).
	
	\item Chương 4. Xây dựng và triển khai quy trình CI/CD: Tập trung vào các bước thực hiện trọng tâm: Xây dựng quy trình CI/CD cho ứng dụng mẫu và tiến hành triển khai quy trình trên các môi trường đã thiết lập.
	
	\item Chương 5. Xây dựng các trường hợp kiểm thử và trình bày kết quả: Trình bày việc xây dựng các trường hợp kiểm thử và phân tích kết quả kiểm thử tự động, đánh giá hiệu quả của quy trình CI/CD đa nền tảng.
	
	\item Kết luận và hướng nghiên cứu: Tổng kết kết quả đạt được và đề xuất hướng cải thiện hoặc mở rộng nghiên cứu trong tương lai.
\end{itemize}

